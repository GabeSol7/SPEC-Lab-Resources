% =========================================================
% COMPREHENSIVE LATEX TEMPLATE FOR UNDERGRADUATE MATHEMATICS
% USC SPEC Lab Resource Guide
% =========================================================

\documentclass[11pt,a4paper]{article}

\usepackage[margin=1in]{geometry} % Standard margin
\usepackage[T1]{fontenc}
\usepackage{mathptmx} % Times New Roman-like font
\usepackage{titlesec} % Section formatting
\usepackage{enumitem} % Better list formatting
\usepackage{booktabs} % For tables
\usepackage{graphicx} % For images
\usepackage{fancyhdr} % Header/Footer
\usepackage{hyperref} % Links
\usepackage{xcolor}

% ----- Define USC/SPEC Lab Colors -----
\definecolor{burgundycardinal}{RGB}{128, 0, 0}
\definecolor{antiquegold}{RGB}{153, 124, 21}
\definecolor{richcharcoal}{RGB}{54, 54, 54}

% ----- Title and Section Formatting -----
\titleformat{\section}{\color{burgundycardinal}\large\bfseries}{}{0em}{}
\titleformat{\subsection}{\color{burgundycardinal}\normalsize\bfseries}{}{0em}{}
\titlespacing*{\section}{0pt}{2ex plus 1ex minus .2ex}{1ex plus .2ex}

% ----- Compact Lists -----
\setlist[itemize]{leftmargin=*, itemsep=2pt, parsep=0pt, topsep=2pt}
\setlist[enumerate]{leftmargin=*, itemsep=2pt, parsep=0pt, topsep=2pt}

% ----- Header and Footer -----
\pagestyle{fancy}
\fancyhf{}
\renewcommand{\headrulewidth}{0pt}
\renewcommand{\footrulewidth}{0.5pt}
\fancyfoot[C]{\textcolor{richcharcoal}{USC SPEC Lab Resource Guide}}
\fancyfoot[R]{\textcolor{burgundycardinal}{\thepage}}

\hypersetup{
    colorlinks=true,
    linkcolor=antiquegold,
    filecolor=antiquegold,
    urlcolor=antiquegold,
    citecolor=antiquegold,
    pdftitle={Comprehensive LaTeX Template for Undergraduate Mathematics},
    pdfauthor={Your Name},
}

% ----- Essential Packages for Mathematics -----
\usepackage{amsmath}    % Advanced math formatting
\usepackage{amssymb}    % Additional math symbols
\usepackage{amsfonts}   % Mathematical fonts
\usepackage{mathtools}  % Extensions for amsmath
\usepackage{amsthm}     % For theorem environments

% ----- Additional Useful Packages -----
\usepackage{tcolorbox}  % For colored boxes
\usepackage{tikz}       % For diagrams
\usepackage{float}      % Better figure positioning

% ----- Custom Theorem Environments -----
\newtheorem{theorem}{Theorem}[section]
\newtheorem{lemma}[theorem]{Lemma}
\newtheorem{proposition}[theorem]{Proposition}
\newtheorem{corollary}[theorem]{Corollary}
\newtheorem{definition}{Definition}[section]
\newtheorem{example}{Example}[section]
\newtheorem{remark}{Remark}[section]

\renewcommand{\qedsymbol}{$\blacksquare$}

% ----- Custom Commands for Math Notation -----
\newcommand{\R}{\mathbb{R}}         % Real numbers
\newcommand{\Z}{\mathbb{Z}}         % Integers
\newcommand{\N}{\mathbb{N}}         % Natural numbers
\newcommand{\Q}{\mathbb{Q}}         % Rational numbers
\newcommand{\C}{\mathbb{C}}         % Complex numbers
\newcommand{\set}[1]{\{#1\}}        % Set notation
\newcommand{\abs}[1]{\left|#1\right|}  % Absolute value
\newcommand{\norm}[1]{\left\|#1\right\|}  % Norm
\newcommand{\innerp}[2]{\langle #1, #2 \rangle}  % Inner product

% ----- Custom Commands for Operators -----
\DeclareMathOperator{\lcm}{lcm}     % Least common multiple
\DeclareMathOperator{\gcd}{gcd}     % Greatest common divisor

% ----- Custom Boxes for Tips, Notes, and Warnings -----
\newtcolorbox{tipbox}{colback=antiquegold!10, colframe=antiquegold, title=Tip}
\newtcolorbox{notebox}{colback=burgundycardinal!10, colframe=burgundycardinal, title=Note}
\newtcolorbox{warningbox}{colback=burgundycardinal!20, colframe=antiquegold, title=Warning}

%%%%%%%%%%%%%%%%%%%%%%%%%%%%%%%%%%%%%%%%%%%%%%%%%%%%%%%%%%%%%%%%%%%%%%
% DOCUMENT CONTENT
%%%%%%%%%%%%%%%%%%%%%%%%%%%%%%%%%%%%%%%%%%%%%%%%%%%%%%%%%%%%%%%%%%%%%%

\begin{document}

\begin{center}
    \vspace*{0.5cm}
    {\color{burgundycardinal} \LARGE \textbf{Comprehensive LaTeX Template for Undergraduate Mathematics}}
    
    \vspace{0.2cm}
    {\color{richcharcoal} \large For Social Science Graduate Programs and Data Science Careers}
    
    \vspace{0.3cm}
    \includegraphics[width=0.3\textwidth]{img/logo.png}
    
    \vspace{0.2cm}
    {\small \textit{A Resource Guide for Undergraduate Students and Research Assistants}}
    
    \vspace{0.2cm}
    {\small \textcolor{richcharcoal}{USC SPEC Lab | \today}}
\end{center}

\vspace{2em}

% =========================================================
% INTRODUCTION SECTION
% =========================================================
\section*{Introduction to This Template}

This document serves as both a template and a guide for typesetting mathematical documents using \LaTeX. It covers basic structures, mathematical notation, theorems, proofs, and other elements commonly used in undergraduate mathematics.

\begin{notebox}
Throughout this template, you'll find explanatory notes like this one that provide additional context and tips for working with \LaTeX.
\end{notebox}

\begin{tipbox}
Most \LaTeX\ editors support auto-completion. Try typing a backslash (\\) followed by the first few letters of a command to see available options.
\end{tipbox}

% =========================================================
% BASIC MATHEMATICS NOTATION
% =========================================================
\section{Basic Mathematical Notation}

This section demonstrates how to typeset common mathematical expressions.

\subsection{Inline vs. Display Mathematics}

Mathematics can be written inline within text using \verb|$...$| or as displayed equations using \verb|\[...\]| or the \verb|equation| environment.

Inline math example: The quadratic formula $x = \frac{-b \pm \sqrt{b^2 - 4ac}}{2a}$ gives solutions to $ax^2 + bx + c = 0$.

Display math example:
\[
x = \frac{-b \pm \sqrt{b^2 - 4ac}}{2a}
\]

Numbered equation:
\begin{equation}
    E = mc^2
\end{equation}

\begin{notebox}
Use inline math for simple expressions that fit naturally within text. Use display math for complex expressions or those you want to emphasize.
\end{notebox}

\subsection{Common Mathematical Symbols}

\subsubsection{Arithmetic Operations}
\begin{align}
    a + b &= \text{addition} \\
    a - b &= \text{subtraction} \\
    a \times b \text{ or } a \cdot b &= \text{multiplication} \\
    a \div b \text{ or } \frac{a}{b} &= \text{division} \\
    a^b &= \text{exponentiation} \\
    \sqrt{a} &= \text{square root} \\
    \sqrt[n]{a} &= \text{nth root}
\end{align}

\subsubsection{Relations}
\begin{align}
    a = b &\quad \text{equality} \\
    a \neq b &\quad \text{inequality} \\
    a < b &\quad \text{less than} \\
    a > b &\quad \text{greater than} \\
    a \leq b &\quad \text{less than or equal to} \\
    a \geq b &\quad \text{greater than or equal to} \\
    a \approx b &\quad \text{approximately equal to} \\
    a \sim b &\quad \text{similar to} \\
    a \propto b &\quad \text{proportional to}
\end{align}

\subsubsection{Set Notation}
\begin{align}
    a \in A &\quad \text{element of} \\
    a \notin A &\quad \text{not an element of} \\
    A \subset B &\quad \text{subset of} \\
    A \subseteq B &\quad \text{subset of or equal to} \\
    A \cup B &\quad \text{union} \\
    A \cap B &\quad \text{intersection} \\
    A \setminus B &\quad \text{set difference} \\
    \emptyset &\quad \text{empty set} \\
    A \times B &\quad \text{Cartesian product}
\end{align}

\subsubsection{Number Sets}
\begin{align}
    \N &= \{1, 2, 3, \ldots\} \quad \text{natural numbers} \\
    \Z &= \{\ldots, -2, -1, 0, 1, 2, \ldots\} \quad \text{integers} \\
    \Q &= \{\frac{p}{q} : p, q \in \Z, q \neq 0\} \quad \text{rational numbers} \\
    \R &= \text{real numbers} \\
    \C &= \{a + bi : a, b \in \R, i^2 = -1\} \quad \text{complex numbers}
\end{align}

\subsubsection{Logical Symbols}
\begin{align}
    p \land q &\quad \text{logical AND} \\
    p \lor q &\quad \text{logical OR} \\
    \neg p &\quad \text{logical NOT} \\
    p \Rightarrow q &\quad \text{implication (if $p$ then $q$)} \\
    p \Leftrightarrow q &\quad \text{equivalence (if and only if)} \\
    \forall x &\quad \text{for all $x$} \\
    \exists x &\quad \text{there exists $x$} \\
    \nexists x &\quad \text{there does not exist $x$}
\end{align}

\begin{tipbox}
Define your own commands for frequently used symbols to save time and ensure consistency. See the preamble of this template for examples.
\end{tipbox}

% =========================================================
% ADVANCED EQUATION FORMATTING
% =========================================================
\section{Advanced Equation Formatting}

\subsection{Multi-line Equations}

The \verb|align| environment is useful for multi-line equations, especially when you want to align them at a specific point (typically equals signs):

\begin{align}
    (a + b)^2 &= (a + b)(a + b) \\
    &= a^2 + ab + ba + b^2 \\
    &= a^2 + 2ab + b^2
\end{align}

\begin{notebox}
Use the \& symbol to specify alignment points in multi-line equations.
\end{notebox}

For equations without alignment points, use the \verb|gather| environment:

\begin{gather}
    f(x) = ax^2 + bx + c \\
    g(x) = dx^3 + ex^2 + fx + g \\
    h(x) = \sin(x) + \cos(x)
\end{gather}

\subsection{Cases}

The \verb|cases| environment is perfect for piecewise functions:

\begin{equation}
    |x| = 
    \begin{cases}
        x & \text{if } x \geq 0 \\
        -x & \text{if } x < 0
    \end{cases}
\end{equation}

\subsection{Matrices}

\LaTeX\ provides several environments for matrices:

\subsubsection{Basic Matrix}
\begin{equation}
    A = 
    \begin{pmatrix}
        a_{11} & a_{12} & a_{13} \\
        a_{21} & a_{22} & a_{23} \\
        a_{31} & a_{32} & a_{33}
    \end{pmatrix}
\end{equation}

\subsubsection{Matrix with Square Brackets}
\begin{equation}
    B = 
    \begin{bmatrix}
        b_{11} & b_{12} \\
        b_{21} & b_{22}
    \end{bmatrix}
\end{equation}

\subsubsection{Matrix with Curly Braces}
\begin{equation}
    C = 
    \begin{Bmatrix}
        c_{11} & c_{12} \\
        c_{21} & c_{22}
    \end{Bmatrix}
\end{equation}

\subsubsection{Matrix with Vertical Bars (Determinant)}
\begin{equation}
    |D| = 
    \begin{vmatrix}
        d_{11} & d_{12} \\
        d_{21} & d_{22}
    \end{vmatrix}
    = d_{11}d_{22} - d_{12}d_{21}
\end{equation}

\begin{warningbox}
Be careful with large matrices! They can easily extend beyond page margins. Consider using smaller font sizes or breaking them into smaller components when necessary.
\end{warningbox}

\subsection{Limits, Sums, and Integrals}

\subsubsection{Limits}
\begin{equation}
    \lim_{x \to 0} \frac{\sin x}{x} = 1
\end{equation}

\subsubsection{Sums}
\begin{equation}
    \sum_{i=1}^{n} i = \frac{n(n+1)}{2}
\end{equation}

\subsubsection{Products}
\begin{equation}
    \prod_{i=1}^{n} i = n!
\end{equation}

\subsubsection{Integrals}
\begin{equation}
    \int_{a}^{b} f(x) \, dx
\end{equation}

\begin{equation}
    \oint_C f(z) \, dz = 2\pi i \sum \text{Res}(f, a_k)
\end{equation}

\begin{equation}
    \iint_D f(x,y) \, dx \, dy
\end{equation}

\begin{equation}
    \iiint_E f(x,y,z) \, dx \, dy \, dz
\end{equation}

\begin{tipbox}
Notice the \verb|\,| command before \verb|dx| in integrals. This adds a small space that improves readability.
\end{tipbox}

% =========================================================
% THEOREMS, PROOFS, AND DEFINITIONS
% =========================================================
\section{Theorems, Proofs, and Definitions}

Mathematics papers and assignments often include theorems, lemmas, proofs, and definitions. \LaTeX\ provides structured environments for these elements.

\subsection{Theorem Example}

\begin{theorem}[Pythagorean Theorem]
    In a right triangle with sides $a$, $b$, and hypotenuse $c$, we have
    \begin{equation}
        a^2 + b^2 = c^2
    \end{equation}
\end{theorem}

\begin{proof}
    Consider a right triangle with sides $a$, $b$, and hypotenuse $c$. We can place this triangle in a coordinate system such that one vertex is at the origin $(0,0)$, another at $(a,0)$, and the third at $(0,b)$.
    
    The distance from $(0,0)$ to $(a,b)$ is given by the distance formula:
    \begin{align}
        c &= \sqrt{(a-0)^2 + (b-0)^2} \\
        c &= \sqrt{a^2 + b^2}
    \end{align}
    
    Squaring both sides:
    \begin{equation}
        c^2 = a^2 + b^2
    \end{equation}
\end{proof}

\begin{lemma}
    Let $a$, $b$, and $c$ be positive real numbers. If $a < b$ and $b < c$, then $a < c$.
\end{lemma}

\begin{proof}
    Since $a < b$, we have $b - a > 0$.
    
    Since $b < c$, we have $c - b > 0$.
    
    Adding these inequalities:
    \begin{align}
        (b - a) + (c - b) &> 0 \\
        c - a &> 0
    \end{align}
    
    Therefore, $a < c$.
\end{proof}

\subsection{Definition Example}

\begin{definition}[Continuous Function]
    A function $f: X \to Y$ between topological spaces is continuous if for every open set $V \subset Y$, the preimage $f^{-1}(V) \subset X$ is open.
\end{definition}

\begin{remark}
    For functions $f: \R \to \R$, this is equivalent to the $\epsilon$-$\delta$ definition: $f$ is continuous at $a$ if for every $\epsilon > 0$, there exists a $\delta > 0$ such that for all $x$ with $|x - a| < \delta$, we have $|f(x) - f(a)| < \epsilon$.
\end{remark}

\begin{example}
    The function $f(x) = x^2$ is continuous on $\R$. To verify this using the $\epsilon$-$\delta$ definition, let $\epsilon > 0$ be given. We need to find $\delta > 0$ such that $|x^2 - a^2| < \epsilon$ whenever $|x - a| < \delta$.
    
    Note that $|x^2 - a^2| = |x + a||x - a|$. If we restrict to $|x - a| < 1$, then $|x| < |a| + 1$, so $|x + a| < 2|a| + 1$.
    
    Thus, we can choose $\delta = \min\{1, \frac{\epsilon}{2|a| + 1}\}$.
\end{example}

\begin{tipbox}
The theorem environments are all numbered automatically. To reference them elsewhere in your document, add a label with \verb|\label{name}| and then refer to it with \verb|\ref{name}|.
\end{tipbox}

% =========================================================
% STRUCTURING HOMEWORK ASSIGNMENTS
% =========================================================
\section{Structuring Homework Assignments}

For most undergraduate mathematics courses, homework assignments follow a simple structure with numbered problems and subproblems.

\subsection{Basic Problem Structure}

\begin{tcolorbox}[title=Problem 1]
    Prove that for any integer $n$, if $n^2$ is even, then $n$ is even.
\end{tcolorbox}

\begin{proof}
    We'll prove the contrapositive: if $n$ is odd, then $n^2$ is odd.
    
    Let $n$ be odd. Then $n = 2k + 1$ for some integer $k$.
    
    Now, $n^2 = (2k + 1)^2 = 4k^2 + 4k + 1 = 2(2k^2 + 2k) + 1$
    
    Since $2k^2 + 2k$ is an integer, $n^2$ is odd by definition.
    
    Therefore, if $n^2$ is even, then $n$ must be even.
\end{proof}

\begin{tcolorbox}[title=Problem 2]
    Solve the following system of equations:
    \begin{align}
        3x + 2y &= 7 \\
        x - y &= 1
    \end{align}
\end{tcolorbox}

\textbf{Solution:}

From the second equation, we have $x = y + 1$. Substituting into the first equation:
\begin{align}
    3(y + 1) + 2y &= 7 \\
    3y + 3 + 2y &= 7 \\
    5y + 3 &= 7 \\
    5y &= 4 \\
    y &= \frac{4}{5}
\end{align}

Now we can find $x$:
\begin{align}
    x &= y + 1 \\
    x &= \frac{4}{5} + 1 \\
    x &= \frac{4}{5} + \frac{5}{5} \\
    x &= \frac{9}{5}
\end{align}

Therefore, the solution is $x = \frac{9}{5}$ and $y = \frac{4}{5}$.

\subsection{Problem with Multiple Parts}

\begin{tcolorbox}[title=Problem 3]
    Consider the function $f(x) = x^3 - 3x + 1$.
    \begin{enumerate}[label=(\alph*)]
        \item Find all critical points of $f$.
        \item Determine whether each critical point is a local maximum, local minimum, or neither.
        \item Find the global maximum and minimum of $f$ on the interval $[-2, 2]$.
    \end{enumerate}
\end{tcolorbox}

\textbf{Solution:}

\begin{enumerate}[label=(\alph*)]
    \item To find the critical points, we calculate $f'(x)$ and set it equal to zero:
    \begin{align}
        f'(x) &= 3x^2 - 3 \\
        3x^2 - 3 &= 0 \\
        3x^2 &= 3 \\
        x^2 &= 1 \\
        x &= \pm 1
    \end{align}
    
    So, the critical points are $x = 1$ and $x = -1$.
    
    \item To determine the nature of these critical points, we compute $f''(x)$:
    \begin{align}
        f''(x) = 6x
    \end{align}
    
    For $x = 1$, we have $f''(1) = 6 > 0$, so $x = 1$ is a local minimum.
    
    For $x = -1$, we have $f''(-1) = -6 < 0$, so $x = -1$ is a local maximum.
    
    \item To find the global extrema on $[-2, 2]$, we evaluate $f$ at the critical points and endpoints:
    \begin{align}
        f(-2) &= (-2)^3 - 3(-2) + 1 = -8 + 6 + 1 = -1 \\
        f(-1) &= (-1)^3 - 3(-1) + 1 = -1 + 3 + 1 = 3 \\
        f(1) &= (1)^3 - 3(1) + 1 = 1 - 3 + 1 = -1 \\
        f(2) &= (2)^3 - 3(2) + 1 = 8 - 6 + 1 = 3
    \end{align}
    
    Therefore, the global maximum is $3$, occurring at $x = -1$ and $x = 2$, and the global minimum is $-1$, occurring at $x = -2$ and $x = 1$.
\end{enumerate}

\begin{tipbox}
The \verb|enumerate| environment with the \verb|enumitem| package gives you great flexibility in formatting lists. Use the \verb|label| option to customize the numbering style.
\end{tipbox}

% =========================================================
% ADDING FIGURES AND DIAGRAMS
% =========================================================
\section{Adding Figures and Diagrams}

\subsection{Including External Images}

To include an external image, use the \verb|\includegraphics| command:

\begin{figure}[H] % The [H] option forces the figure to appear exactly here
    \centering
    \includegraphics[width=0.6\textwidth]{example-image} % Replace with your image file
    \caption{Example figure caption explaining the image context and significance.}
    \label{fig:example}
\end{figure}

You can reference figures using their labels: See Figure \ref{fig:example}.

\begin{notebox}
The \verb|example-image| is a built-in placeholder image provided by the \verb|graphicx| package. Replace it with your actual image file.
\end{notebox}

\subsection{Creating Diagrams with TikZ}

TikZ is a powerful package for creating diagrams directly within LaTeX:

\begin{figure}[H]
    \centering
    \begin{tikzpicture}
        % Coordinate axes
        \draw[-latex] (-0.5,0) -- (4.5,0) node[right] {$x$};
        \draw[-latex] (0,-0.5) -- (0,4.5) node[above] {$y$};
        
        % Function plot
        \draw[domain=0:4, smooth, variable=\x, blue, thick] plot ({\x}, {\x*\x/4});
        
        % Point on the curve
        \filldraw[red] (2,1) circle (2pt) node[above right] {$(2,1)$};
        
        % Tangent line at x=2
        \draw[red, dashed] (0,0) -- (4,2);
        
        % Label for the function
        \node at (3,3) {$y = \frac{x^2}{4}$};
    \end{tikzpicture}
    \caption{A parabola with its tangent line at $x = 2$.}
    \label{fig:parabola}
\end{figure}

\begin{tipbox}
TikZ has a steep learning curve but is extremely versatile. Start with simple diagrams and gradually explore more commands as needed.
\end{tipbox}

% =========================================================
% BIBLIOGRAPHY AND CITATIONS
% =========================================================
\section{Bibliography and Citations}

In more advanced mathematics writing, you'll often need to cite references. Here's a basic example using the \verb|thebibliography| environment:

To cite a reference in your text, use the \verb|\cite| command:

As shown by Knuth \cite{knuth1984texbook}, \LaTeX\ is a powerful typesetting system for mathematics.

\begin{thebibliography}{9}
    \bibitem{knuth1984texbook} Knuth, D. E. (1984). The TeXbook. Addison-Wesley.
    \bibitem{lamport1994latex} Lamport, L. (1994). \LaTeX: A Document Preparation System. Addison-Wesley.
    \bibitem{stewart2015calculus} Stewart, J. (2015). Calculus: Early Transcendentals. Cengage Learning.
\end{thebibliography}

\begin{notebox}
For more complex documents, consider using BibTeX or BibLaTeX for managing references.
\end{notebox}

% =========================================================
% COMMON MISTAKES AND TIPS
% =========================================================
\section{Common Mistakes and Tips}

\subsection{Spacing in Math Mode}

\LaTeX\ automatically handles most spacing in mathematical expressions, but sometimes manual adjustments are needed:

\begin{itemize}
    \item Use \verb|\,| for a thin space: $f(x)\,dx$
    \item Use \verb|\;| for a medium space: $A\;B$
    \item Use \verb|\quad| for a larger space: $A\quad B$
    \item Use \verb|\qquad| for an even larger space: $A\qquad B$
    \item Use \verb|\!| for a negative thin space (bringing symbols closer): $\int\!\!\!\int$
\end{itemize}

\subsection{Braces and Brackets}

Use \verb|\left| and \verb|\right| to create automatically sized delimiters:

\begin{equation}
    \left( \frac{a}{b} \right) \quad
    \left[ \sum_{i=1}^{n} i^2 \right] \quad
    \left\{ \frac{1}{1-x} \right\} \quad
    \left| \frac{x}{y} \right| \quad
    \left\| \vec{v} \right\|
\end{equation}

\subsection{Text in Math Mode}

Use \verb|\text{}| for words within mathematical expressions:

\begin{equation}
    f(x) = 
    \begin{cases}
        x^2 & \text{if $x \geq 0$} \\
        -x^2 & \text{if $x < 0$}
    \end{cases}
\end{equation}

\begin{warningbox}
Never use normal text directly in math mode! Always wrap it with \verb|\text{}| to ensure proper spacing and font consistency.
\end{warningbox}

\subsection{Multi-letter Variables and Functions}

Standard functions should use upright font, while variables use italic:

\begin{itemize}
    \item Correct: $\sin(x)$, $\log(y)$, $\lim_{x \to 0} f(x)$
    \item Incorrect: $sin(x)$, $log(y)$
\end{itemize}

Common mathematical functions have predefined commands:
\begin{itemize}
    \item Trigonometric: \verb|\sin|, \verb|\cos|, \verb|\tan|, \verb|\csc|, \verb|\sec|, \verb|\cot|
    \item Inverse trigonometric: \verb|\arcsin|, \verb|\arccos|, \verb|\arctan|
    \item Hyperbolic: \verb|\sinh|, \verb|\cosh|, \verb|\tanh|
    \item Logarithmic and exponential: \verb|\log|, \verb|\ln|, \verb|\exp|
    \item Others: \verb|\lim|, \verb|\max|, \verb|\min|, \verb|\sup|, \verb|\inf|, \verb|\det|, \verb|\gcd|
\end{itemize}

For functions not predefined, use \verb|\operatorname{name}| or define your own with \verb|\DeclareMathOperator|.

% =========================================================
% CONCLUSION
% =========================================================
\section{Conclusion}

This template provides a comprehensive starting point for typesetting mathematical documents using \LaTeX. As you become more familiar with the system, you'll discover additional packages and commands that suit your specific needs.

Remember that learning \LaTeX\ is an incremental process. Start with basic documents and gradually incorporate more advanced features as you become comfortable.

\begin{tipbox}
Practice regularly with small documents before attempting larger projects. When encountering errors, read the error messages carefully—they often provide helpful information for troubleshooting.
\end{tipbox}

\begin{notebox}
Online resources like the \LaTeX\ Stack Exchange (\url{https://tex.stackexchange.com/}) and the Overleaf Documentation (\url{https://www.overleaf.com/learn}) are invaluable for learning more advanced techniques and solving specific problems.
\end{notebox}

% End of document
\end{document}
